\documentclass[journal]{vgtc}                % final (journal style)
%\documentclass[review,journal]{vgtc}         % review (journal style)
%\documentclass[widereview]{vgtc}             % wide-spaced review
%\documentclass[preprint,journal]{vgtc}       % preprint (journal style)

%% Uncomment one of the lines above depending on where your paper is
%% in the conference process. ``review'' and ``widereview'' are for review
%% submission, ``preprint'' is for pre-publication, and the final version
%% doesn't use a specific qualifier.

%% Please use one of the ``review'' options in combination with the
%% assigned online id (see below) ONLY if your paper uses a double blind
%% review process. Some conferences, like IEEE Vis and InfoVis, have NOT
%% in the past.

%% Please note that the use of figures other than the optional teaser is not permitted on the first page
%% of the journal version.  Figures should begin on the second page and be
%% in CMYK or Grey scale format, otherwise, colour shifting may occur
%% during the printing process.  Papers submitted with figures other than the optional teaser on the
%% first page will be refused. Also, the teaser figure should only have the
%% width of the abstract as the template enforces it.

%% These few lines make a distinction between latex and pdflatex calls and they
%% bring in essential packages for graphics and font handling.
%% Note that due to the \DeclareGraphicsExtensions{} call it is no longer necessary
%% to provide the the path and extension of a graphics file:
%% \includegraphics{diamondrule} is completely sufficient.
%%
\ifpdf%                                % if we use pdflatex
  \pdfoutput=1\relax                   % create PDFs from pdfLaTeX
  \pdfcompresslevel=9                  % PDF Compression
  \pdfoptionpdfminorversion=7          % create PDF 1.7
  \ExecuteOptions{pdftex}
  \usepackage{graphicx}                % allow us to embed graphics files
  \DeclareGraphicsExtensions{.pdf,.png,.jpg,.jpeg} % for pdflatex we expect .pdf, .png, or .jpg files
\else%                                 % else we use pure latex
  \ExecuteOptions{dvips}
  \usepackage{graphicx}                % allow us to embed graphics files
  \DeclareGraphicsExtensions{.eps}     % for pure latex we expect eps files
\fi%

%% it is recomended to use ``\autoref{sec:bla}'' instead of ``Fig.~\ref{sec:bla}''
\graphicspath{{figures/}{pictures/}{images/}{./}} % where to search for the images

\usepackage{microtype}                 % use micro-typography (slightly more compact, better to read)
\PassOptionsToPackage{warn}{textcomp}  % to address font issues with \textrightarrow
\usepackage{textcomp}                  % use better special symbols
\usepackage{mathptmx}                  % use matching math font
\usepackage{times}                     % we use Times as the main font
\renewcommand*\ttdefault{txtt}         % a nicer typewriter font
\usepackage{cite}                      % needed to automatically sort the references
\usepackage{tabu}                      % only used for the table example
\usepackage{booktabs}                  % only used for the table example
%% We encourage the use of mathptmx for consistent usage of times font
%% throughout the proceedings. However, if you encounter conflicts
%% with other math-related packages, you may want to disable it.

%% In preprint mode you may define your own headline.
%\preprinttext{To appear in IEEE Transactions on Visualization and Computer Graphics.}

%% If you are submitting a paper to a conference for review with a double
%% blind reviewing process, please replace the value ``0'' below with your
%% OnlineID. Otherwise, you may safely leave it at ``0''.
\onlineid{0}

%% declare the category of your paper, only shown in review mode
\vgtccategory{Research}
%% please declare the paper type of your paper to help reviewers, only shown in review mode
%% choices:
%% * algorithm/technique
%% * application/design study
%% * evaluation
%% * system
%% * theory/model
\vgtcpapertype{application/design study}

%% Paper title.
\title{Title of your work}

%% This is how authors are specified in the journal style

%% indicate IEEE Member or Student Member in form indicated below
\author{Gabriela Slavova, Henrique Dias, Nimo Beeren}
\authorfooter{
%% insert punctuation at end of each item
\item
Gabriela Slavova, Henrique Dias and Nimo Beeren are with Eindhoven University of
Technology. E-mail: \{g.slavova, h.a.coelho.dias, n.beeren\}@student.tue.nl.
}

%other entries to be set up for journal
%\shortauthortitle{Biv \MakeLowercase{\textit{et al.}}: Global Illumination for Fun and Profit}
%\shortauthortitle{Firstauthor \MakeLowercase{\textit{et al.}}: Paper Title}

%% Abstract section.
\abstract{Abstract text%
} % end of abstract

%% Keywords that describe your work. Will show as 'Index Terms' in journal
%% please capitalize first letter and insert punctuation after last keyword
\keywords{keyword 1, keyword 2, \ldots}

%% ACM Computing Classification System (CCS). 
%% See <http://www.acm.org/class/1998/> for details.
%% The ``\CCScat'' command takes four arguments.

%\CCScatlist{ % not used in journal version
% \CCScat{K.6.1}{Management of Computing and Information Systems}%
%{Project and People Management}{Life Cycle};
% \CCScat{K.7.m}{The Computing Profession}{Miscellaneous}{Ethics}
%}

%% Uncomment below to include a teaser figure.
%\teaser{
%  \centering
%  \includegraphics[width=\linewidth]{CypressView}
%  \caption{In the Clouds: Vancouver from Cypress Mountain. Note that the teaser may not be wider than the abstract block.}
%	\label{fig:teaser}
%}

%% Uncomment below to disable the manuscript note
%\renewcommand{\manuscriptnotetxt}{}

%% Copyright space is enabled by default as required by guidelines.
%% It is disabled by the 'review' option or via the following command:
% \nocopyrightspace

\vgtcinsertpkg

%%%%%%%%%%%%%%%%%%%%%%%%%%%%%%%%%%%%%%%%%%%%%%%%%%%%%%%%%%%%%%%%
%%%%%%%%%%%%%%%%%%%%%% START OF THE PAPER %%%%%%%%%%%%%%%%%%%%%%
%%%%%%%%%%%%%%%%%%%%%%%%%%%%%%%%%%%%%%%%%%%%%%%%%%%%%%%%%%%%%%%%

\begin{document}

%% The ``\maketitle'' command must be the first command after the
%% ``\begin{document}'' command. It prepares and prints the title block.

%% the only exception to this rule is the \firstsection command
\firstsection{Introduction}

\maketitle

% Briefly introduce the topic that you are going to treat in your work and why it is relevant.

In recent times, it has become increasingly clear to most people that climate change is affecting life on earth. As temperatures rise globally, many animals are forced to adapt to their environment \cite{TODO}. Among the best-documented instances of such adaptation are changes in bird migration times \cite{TODO}. This paper presents an interactive visualization with the goal to communicate these findings.

\subsection{Problem Description}

% Describe the general goal of your work. Ideally, this section should depict the main domain/topic of your work, what your visual approach will achieve overall and how complex the addressed problem is.

Despite evidence presented in numerous studies, it has proved to be difficult to communicate the effects of global warming to regular people. However, some effects can already be observed in everyday life, without the need for much expertise. One such effect is the change in the time of the year during which migratory birds can be observed in their breeding habitat. As global temperatures increased, birds have been shown to remain in their breeding habitat for longer. Using a visual approach, we intend to show this correlation in a way that is interpretable to experts and non-experts alike.

Both bird migration and surface temperature changes are cyclic processes with a period of one year. In order to infer trends in these processes, we have to consider data from multiple years, or even decades. However, because of the cyclic nature, it is difficult to directly compare one year to another. One of the challenges of this problem is to visualize trends over the long term, without obscuring the patterns that occur every year.

Another thing that these processes have in common is that they are both geographic. It will be essential to visualize both in such a way that one does not obscure the other, while their correlation is still made clear.

\section{Data Analysis}

\subsection{Domain Data Specification}

Here you should describe the data that you will use, its main characteristics and how it relates to the domain you are treating.

\subsection{Data Abstraction: What}

After a short description on the data has been provided, you will present a data abstraction (\emph{What}) describing the type of data that you cope with: tabular, network, types of attributes (e.g, quantitative, sequential), spatial, etc.

\section{Task Analysis}

\subsection{Domain Specific Tasks}

In this section, you must analyze the possible visualization tasks that an expert in the chosen domain would be interested in. Ideally, you would present a set of tasks and questions that are intimately related to the chosen domain.

\subsection{Task Abstraction: Why}

Once you have defined a set of domain specific tasks, you must generalize them to be domain independent to be able to use the principles from visualization. In this sense, we propose you follow the principles described in Munzner's book \cite{munzner2014visualization} and described in the lectures.

\section{Visualization and Interaction Design: How}

Describe and motivate your choices for the visual encodings and interaction design (\emph{How}). You should not think about implementation yet, but rather what visual encodings fit best to address the task/data results presented earlier. You must justify your choices based on the design principles described in the course. Here is where you explicitly link your tasks, questions and data to the design, and argue why the chosen design will enable users to perform the tasks and answer their questions. Explain the main interactions provided in your solution.

\section{Realization}

After you have a visualization design, you should elaborate on what you have implemented and what compromises were made. For instance, your ideal solution may consider edge bundling \cite{holten2006hierarchical} but you could not implement it in your final solution. In this section, you should also mention what framework/technology you used for your implementation and justify why you chose it.

\section{Use Cases}

Describe the analysis of the data and how your application was used to perform the analysis. Here you demonstrate how to use your application to perform the tasks and answer the questions as an evaluation of your design and implementation.


\section{Discussion and Conclusion}

Finally, you will reflect on what you have achieved. For instance, this section should discuss whether you managed to do what you initially planned, whether your initial choices worked well or not, things that you discovered that were not correct, etc. Conclusions about your whole work would be provided.

\bibliographystyle{abbrv}
\bibliography{main}
\end{document}

