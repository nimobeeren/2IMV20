\documentclass[journal]{vgtc}                % final (journal style)
%\documentclass[review,journal]{vgtc}         % review (journal style)
%\documentclass[widereview]{vgtc}             % wide-spaced review
%\documentclass[preprint,journal]{vgtc}       % preprint (journal style)

%% Uncomment one of the lines above depending on where your paper is
%% in the conference process. ``review'' and ``widereview'' are for review
%% submission, ``preprint'' is for pre-publication, and the final version
%% doesn't use a specific qualifier.

%% Please use one of the ``review'' options in combination with the
%% assigned online id (see below) ONLY if your paper uses a double blind
%% review process. Some conferences, like IEEE Vis and InfoVis, have NOT
%% in the past.

%% Please note that the use of figures other than the optional teaser is not permitted on the first page
%% of the journal version.  Figures should begin on the second page and be
%% in CMYK or Grey scale format, otherwise, colour shifting may occur
%% during the printing process.  Papers submitted with figures other than the optional teaser on the
%% first page will be refused. Also, the teaser figure should only have the
%% width of the abstract as the template enforces it.

%% These few lines make a distinction between latex and pdflatex calls and they
%% bring in essential packages for graphics and font handling.
%% Note that due to the \DeclareGraphicsExtensions{} call it is no longer necessary
%% to provide the the path and extension of a graphics file:
%% \includegraphics{diamondrule} is completely sufficient.
%%
\ifpdf%                                % if we use pdflatex
  \pdfoutput=1\relax                   % create PDFs from pdfLaTeX
  \pdfcompresslevel=9                  % PDF Compression
  \pdfoptionpdfminorversion=7          % create PDF 1.7
  \ExecuteOptions{pdftex}
  \usepackage{graphicx}                % allow us to embed graphics files
  \DeclareGraphicsExtensions{.pdf,.png,.jpg,.jpeg} % for pdflatex we expect .pdf, .png, or .jpg files
\else%                                 % else we use pure latex
  \ExecuteOptions{dvips}
  \usepackage{graphicx}                % allow us to embed graphics files
  \DeclareGraphicsExtensions{.eps}     % for pure latex we expect eps files
\fi%

%% it is recommended to use ``\autoref{sec:bla}'' instead of ``Fig.~\ref{sec:bla}''
\graphicspath{{figures/}{pictures/}{images/}{./}} % where to search for the images

\usepackage{microtype}                 % use micro-typography (slightly more compact, better to read)
\PassOptionsToPackage{warn}{textcomp}  % to address font issues with \textrightarrow
\usepackage{textcomp}                  % use better special symbols
\usepackage{mathptmx}                  % use matching math font
\usepackage{times}                     % we use Times as the main font
\renewcommand*\ttdefault{txtt}         % a nicer typewriter font
\usepackage{cite}                      % needed to automatically sort the references
\usepackage{tabu}                      % only used for the table example
\usepackage{booktabs}                  % only used for the table example
%% We encourage the use of mathptmx for consistent usage of times font
%% throughout the proceedings. However, if you encounter conflicts
%% with other math-related packages, you may want to disable it.

%% In preprint mode you may define your own headline.
%\preprinttext{To appear in IEEE Transactions on Visualization and Computer Graphics.}

%% If you are submitting a paper to a conference for review with a double
%% blind reviewing process, please replace the value ``0'' below with your
%% OnlineID. Otherwise, you may safely leave it at ``0''.
\onlineid{0}

%% declare the category of your paper, only shown in review mode
\vgtccategory{Research}
%% please declare the paper type of your paper to help reviewers, only shown in review mode
%% choices:
%% * algorithm/technique
%% * application/design study
%% * evaluation
%% * system
%% * theory/model
\vgtcpapertype{application/design study}

%% Paper title.
\title{How climate change affects bird migration}

%% This is how authors are specified in the journal style

%% indicate IEEE Member or Student Member in form indicated below
\author{Gabriela Slavova, Henrique Dias, Nimo Beeren}
\authorfooter{
%% insert punctuation at end of each item
\item
Gabriela Slavova, Henrique Dias and Nimo Beeren are with Eindhoven University of
Technology. E-mail: \{g.slavova, h.a.coelho.dias, n.beeren\}@student.tue.nl.
}

%other entries to be set up for journal
%\shortauthortitle{Biv \MakeLowercase{\textit{et al.}}: Global Illumination for Fun and Profit}
%\shortauthortitle{Firstauthor \MakeLowercase{\textit{et al.}}: Paper Title}

%% Abstract section.
\abstract{Abstract text%
} % end of abstract

%% Keywords that describe your work. Will show as 'Index Terms' in journal
%% please capitalize first letter and insert punctuation after last keyword
\keywords{Climate change, bird migration.}

%% ACM Computing Classification System (CCS). 
%% See <http://www.acm.org/class/1998/> for details.
%% The ``\CCScat'' command takes four arguments.

%\CCScatlist{ % not used in journal version
% \CCScat{K.6.1}{Management of Computing and Information Systems}%
%{Project and People Management}{Life Cycle};
% \CCScat{K.7.m}{The Computing Profession}{Miscellaneous}{Ethics}
%}

%% Uncomment below to include a teaser figure.
%\teaser{
%  \centering
%  \includegraphics[width=\linewidth]{CypressView}
%  \caption{In the Clouds: Vancouver from Cypress Mountain. Note that the teaser may not be wider than the abstract block.}
%	\label{fig:teaser}
%}

%% Uncomment below to disable the manuscript note
%\renewcommand{\manuscriptnotetxt}{}

%% Copyright space is enabled by default as required by guidelines.
%% It is disabled by the 'review' option or via the following command:
% \nocopyrightspace

\vgtcinsertpkg

%%%%%%%%%%%%%%%%%%%%%%%%%%%%%%%%%%%%%%%%%%%%%%%%%%%%%%%%%%%%%%%%
%%%%%%%%%%%%%%%%%%%%%% START OF THE PAPER %%%%%%%%%%%%%%%%%%%%%%
%%%%%%%%%%%%%%%%%%%%%%%%%%%%%%%%%%%%%%%%%%%%%%%%%%%%%%%%%%%%%%%%

\begin{document}

%% The ``\maketitle'' command must be the first command after the
%% ``\begin{document}'' command. It prepares and prints the title block.

%% the only exception to this rule is the \firstsection command
\firstsection{Introduction}

\maketitle

In recent times, it has become increasingly clear to most people that climate change is affecting life on Earth. As temperatures rise globally, many animals are forced to adapt to their environment \cite{parmesan2007pheno,gaughan2009domestic,root2003fingerprints}. Among the best-documented instances of such adaptation are changes in bird migration times \cite{miller2008bird,visser2008climate,jenni2003timing}. This paper presents an interactive visualization with the goal to communicate these findings.

\subsection{Problem Description}

Despite evidence presented in numerous studies \cite{solomon2007climate,parry2007climate}, it has proved to be difficult to communicate the effects of climate change to people \cite{lee2015predictors,brulle2012shifting,moser2011communicating}. We believe that visualizing effects that can be observed in everyday life is an important way of fostering public engagement. One such effect is the change in the time of the year during which migratory birds can be observed in their breeding habitat. As global temperatures increased, birds have been migrating to their breeding habitats earlier in the year \cite{cotton2003avian,marra2005influence,jenni2003timing}. Using a visual approach, we intend to show this correlation in a way that is interpretable to experts and non-experts alike.

Both bird migration and surface temperature changes are cyclic processes with an annual period. In order to infer trends in these processes, we should consider data from multiple years, or even decades. However, because of the cyclic nature, it is difficult to directly compare one year to another. One of the challenges of this problem is to visualize trends over the long term, without obscuring the patterns that occur every year.

Another attribute that these processes have in common is that they are both geographic. That is, both produce data that is associated with a particular place on Earth \cite{iso2014geo}. It is essential to visualize both in such a way that one does not obscure the other, while their correlation is still visible.

\section{Data Analysis}

\subsection{Domain Data Specification}

In order to visualize the problem, we need two different datasets: one that includes information about birds migration and one including temperature changes over the years.

For the first needed dataset we have chosen the eBird Basic Dataset \cite{ebird2020data}. The eBird citizen science project is unmatched in scale, containing over 600 million observations, including nearly every bird species on Earth \cite{strimas2020ebird}. Submissions are individually reviewed to ensure accuracy of information. As people are contributing to this dataset continuously, the information inside is up-to-date, but the record also goes many years back.

For the second, we have chosen the GISTEMP dataset \cite{gistemp} by NASA, which provides the temperature data from 1880 up to 2020 in a regular 2° by 2° grid. The main advantage of this dataset compared to other temperature datasets we found is that this one uses the temperature anomalies. These anomalies are relative to a 30-year period between 1951 and 1980. By using anomalies instead of absolute values \cite{gistempanomalies}, it is easier to see how much colder or warmer a certain place is. 

\subsection{Data Abstraction: What}

The eBird Basic Dataset \cite{ebird2020data} consists of a large static table containing items that each correspond to a particular observation of a bird or a group of birds. For each observation, a number of attributes are provided, though we are only interested in the following subset:

\paragraph{Species} A categorical attribute represented by the scientific name, e.g. \textit{Anas platyrhynchos}. The eBird taxonomy\cite{ebird2019taxonomy} specifies exactly what is meant with these names.

\paragraph{Date} An ordered, quantitative and diverging attribute in year-month-day format (yyyy-mm-dd).

\paragraph{Location} A spatial attribute consisting of a vector with components latitude and longitude. Each component is itself an ordered, quantitive and cyclic attribute represented as decimal degrees, where positive values denote N/W and negative values denote S/E for latitude/longitude respectively. The precision varies between observations, but is typically 0.001 degrees or more precise (which is less than 1 km on Earth's surface).

\paragraph{Count} An ordered, quantitative and diverging attribute represented as an integer. Since participants are allowed to group multiple birds in a single observation, the value can be 1 or greater.

\vspace{2mm}

We perform a number of transformations on the original dataset for the purpose of our visualization. First, we limit ourselves to a single species, since different species exhibit significantly different migratory behavior. We chose the Barn Swallow (\emph{Hirundo rustica}) because it is a relatively common migratory bird, and it typically migrates over a large distance \cite{turner1989swallow}. Next, we restrict the dataset geographically to Europe and Africa. These transformations allow us to focus on a general population of birds. Finally, we exclude observations outside the years 1950-2020, since that matches the domain of the temperature dataset we use.

\vspace{2mm}

The GISTEMP dataset \cite{gistemp} consists of a large static multidimensional scalar field, whose dimensions are the latitude, the longitude and the date. For each point in space-time, a temperature anomaly value is available. The description of each dimension and the value:

\paragraph{Latitude} An ordered quantitative cyclic attribute represented as decimal degrees, varying from -90.0 to 90.0.

\paragraph{Longitude} An ordered quantitative cyclic attribute represented as decimal degrees, varying from -180.0 to 180.0.

\paragraph{Temperature Anomaly} A quantitive attribute represented in degrees Celcius.

\paragraph{Date} An ordered, quantitive and diverging attribute in year-month format (yyyy-month).

\section{Task Analysis}

\subsection{Domain Specific Tasks}
% In this section, you must analyze the possible visualization tasks that an expert in the chosen domain would be interested in. Ideally, you would present a set of tasks and questions that are intimately related to the chosen domain.
This visualization is designed to be used by a large variety of specialists and non-specialists alike. An ornithologist could study the migration of the Barn Swallow using the provided visualization. It is also suitable for the needs of climate  change activists who wants to prove a point about a real prolem or a curious person without specific domain knowledge who wants to see the effects of climate change on the nature. The tasks that  these people could undertake are described below.

\paragraph{Task 1: Analyze correlation between temperature change and the timing of bird migration} One of the observations that can be achieved with this visualization is showing the migration of birds in relation to the change in temperature over time. This can be done by choosing a bird species and comparing timing of their migration with respect to the temperatures for the specific time and the same migration in a different time span before or after the chosen one.  The question that would be answered by this task is: how are temperature and the timing of bird migration correlated?

\paragraph{Task 2: Characterize the migration behavior of a species} Another thing that is possible to be visualized is the migration pattern of a specific type of bird over the course of a year. This can be done be choosing a bird species and following its migration behavior through a chosen year. The question that would be answered by this task is: what is the general migration behavior pattern for a particular bird species?

\paragraph{Task 3: Identify species that respond most to temperature change} Different types of birds respond differently to a change in temperature. The tool can help visualize which types of birds are affected by the rising temperatures more than others. This can be done by comparing the migration behavior in relation to the temperature over time, for a set of species. The question that would be answered by this task is: which bird species respond to temperature change the most?

\subsection{Task Abstraction: Why}

% Once you have defined a set of domain specific tasks, you must generalize them to be domain independent to be able to use the principles from visualization. In this sense, we propose you follow the principles described in Munzner's book \cite{munzner2014visualization} and described in the lectures.

\paragraph{Task 1: Analyze correlation between temperature change and the timing of bird migration} The high-level analyze action that this task performs is consume and more precisely discover. This is because discover is related to the act of verifying or disproving of a hypothesis which is exactly the intention behind this task. The mid-level search action is explore as the user doesn't know the location of the target or what the target itself is. The low-level query action is compare as the user will compare different data. The target of this task is a correlation.

\paragraph{Task 2: Characterize the migration behavior of a species} The high-level analyze action in this task is discover, because discover is also generally related to the act of observing without knowing specifically the end result. The mid-level search action is explore as the location of the target and the target itself are unknown. The low-level query action is summarize because this task aims to return general overview of the patterns in the data. The target of this task is also a trend.

\paragraph{Task 3: Identify species that respond most to temperature change} The high-level analyze action performed in this task is also discover. The mid-level search action is explore. The low-level query action is identify. The target of this task is outliers.

\section{Visualization and Interaction Design: How}

% Describe and motivate your choices for the visual encodings and interaction design (\emph{How}). You should not think about implementation yet, but rather what visual encodings fit best to address the task/data results presented earlier. You must justify your choices based on the design principles described in the course. Here is where you explicitly link your tasks, questions and data to the design, and argue why the chosen design will enable users to perform the tasks and answer their questions. Explain the main interactions provided in your solution.

All identified visualization tasks revolve around the analysis of patterns over time in geospatial data. It is therefore critical that our visualization encodes the spatial as well as the temporal dimension in an effective way.

One of the most common encodings of geospatial data is the map. Maps are unique in their ability to provide an overview of an area, while preserving attributes such as distance, area, and distribution \cite{kraak2020cartography}. Another powerful communicative aspect of maps is that they are recognizable. Given a map of the world, most people are able to pick out the place where they live, and thereby select data that is most relevant to them.

We chose to visually encode the bird observations as a heatmap \cite{munzner2014visualization} made up of a grid of equal-sized cells, where the color of each cell corresponds to the total count of all observations within the bounds of that cell. An alternative encoding for bird observations is the scatterplot, where individual observations are represented by a \emph{glyph}. However, a common problem with scatterplots is \emph{overplotting} \cite{micallef2017scatter}, making it difficult to distinguish observations from eachother in dense areas. Another issue with this approach is that it suggests high spatial precision of observations, while this may not always be the case.

Considering that the \emph{count} attribute of the bird data is quantitative and sequentual, we chose a sequential color map, using only the luminance channel. We use a fixed green hue, since the human eye can most easily distinguish shades of green \cite{robinson1984fluorescent}. The saturation channel is used to indicate missing data, by means of a gray color.

For the temperature data, we apply a similar approach as we did for the birds. The key difference is that, conceptually, temperature is a continuous field. Hence, there is no need to discretize the spatial attributes to form a grid. Instead, we draw a scalar field, which could be considered as a heatmap with infinitely small cells. Now, we use the temperature anomaly value to determine the color of each point.

To choose a color map for temperature, it is important to see that the temperature anomaly is a relative measurement and therefore is ordered in diverging directions. The scale ranges from cold to warm, and there is a natural midpoint of 0 °C. As blue is often associated with cold and red with warm, we use these two hues for the respective endpoints of the scale. Again, gray is used to indicate areas of missing data. Another commonly used color map for temperature is the rainbow map. The main problems for us are that this color map does not have a natural midpoint, and that it produces \emph{bands} that do not exist in the data \cite{borland2007rainbow}.

Another decision to be made is what map projection to use. The stereotypical candidate is the Mercator projection \cite{monmonier2010mercator}. A useful property of this projection is that any straight line represents an actual compass bearing, for example up is always north \cite{kennedy2000projections}. However, it also significantly distorts areas further from the equator. This would make some of our grid cells appear larger than others, even though they are of equal area. To combat these issues, we use the Equirectangular projection \cite{miller1949equi}. This projection preserves distance as well as north/east/south/west directions \cite{kennedy2000projections}. Finally, we also draw the coastlines to provide context to the map.

Since the bird and temperature data cover the same spatial and temporal domain, we could consider displaying both in a single map. One way to achieve this is to use a bivariate color map \cite{brewer99color,rheingans2000task}. However, to judge correlation between the two variables, the color must be decomposed into two components, which can be challenging \cite{rheingans2000task}. For this reason, we choose to apply the \emph{small multiples} principle \cite{munzner2014visualization} by drawing two separate maps.

A dimension that we have not yet discussed is time. A traditional method of visualizing change in a map over time is to draw multiple small maps, each representing a different point in time \cite{bertin1983semiology}. This approach is most suitable when only a few points in time need to be visualized. However, the process we are studying spans multiple decades, and we would like to move through this process in small steps in order to analyze correlation. Drawing a static map for each of these steps would would take up too much space to be practical. An alternative that is particularly suited to showing change over time is animation \cite{peterson1995interactive}. In particular, we use an interactive approach that lets the user play and pause the animation, as well as manipulate time manually.

The choice of time scale depends on the task at hand. We therefore allow the time scale to be set to either an annual or monthly mode. When we want to analyze the response of birds to climate change, we need to consider multiple years, if not decades. In this context, the annual patterns that occur in the data are only noise. For this reason, we show a single month for each year. Since a change in the onset of breading season is best supported \cite{marra2005influence,cotton2003avian,jenni2003timing}, we choose March as the default month, although this can be configured. Because the timing of migration differs between bird species \cite{jenni2003timing}, this choice can be significant. The availability of data should also be considered. When we instead want to characterize the patterns in bird migration and temperature change within a single year, we can change to a monthly mode. Now, the animation loops over the months, while the year is fixed (although it can be changed manually).

The maps provide a detailed view of the distribution of birds and temperature at a particular moment in time. Using animation, we start to get an idea of the change over time. However, comparing two points in time requires the user to remember an old state and compare it to the new one. Following the principle of \emph{eyes beat memory} \cite{munzner2014visualization}, we would prefer to show the entire timespan at once. In other words, we want to visually encode the dimension of time, without relying on animation. This increases the total amount of dimensions to four: latitude, longitude, time, and the dependent variable (bird frequency or surface temperature). To make the visualization interpretable, we need to reduce the number of dimensions.

The phenomenon of bird migration can be summarized as a population of birds that move from their southern wintering areas to the northern breeding grounds and back again. The dimension of longitude is less important, because most birds move less along the east-west axis \cite{alerstam1993bird}. In addition, we are less interested in the distribution of the entire population, but rather in their average position. This reduces the number of dimensions to just two: mean latitude and time. To get a general idea of the change in temperature over time, we take the mean anomaly of the entire area under investigation. We are left with the dimensions of mean temperature anomaly and time.

Because we have just two dimensions for each attribute, we have a multitude of options for our visual encoding. We found the line chart to be the most suitable for our use case. In particular, we choose to draw both attributes in a single chart, using a double y-axis. While this makes it more difficult to read individual values from the chart, we can more easily analyze the similarity of the two lines, which is our main goal. If we were to drop the temporal dimension as well, we could have used a scatterplot. This encoding is particularly suited to analyzing the correlation between two variables such as temperature and bird position. However, by ignoring time, we lose sight of any trends in the data, and limits our ability to predict the future. Considering our goal is to show where the path of climate change leads, the scatterplot seems less than ideal.

% write about how percentage is derived from absolute bird count
% write about interaction between maps and line chart
% idea for discussion: investigate temperature at mean bird location rather than overall mean

\section{Realization}

% After you have a visualization design, you should elaborate on what you have implemented and what compromises were made. For instance, your ideal solution may consider edge bundling \cite{holten2006hierarchical} but you could not implement it in your final solution. In this section, you should also mention what framework/technology you used for your implementation and justify why you chose it.

\section{Use Cases}

% Describe the analysis of the data and how your application was used to perform the analysis. Here you demonstrate how to use your application to perform the tasks and answer the questions as an evaluation of your design and implementation.

\section{Discussion and Conclusion}

% Finally, you will reflect on what you have achieved. For instance, this section should discuss whether you managed to do what you initially planned, whether your initial choices worked well or not, things that you discovered that were not correct, etc. Conclusions about your whole work would be provided.

\bibliographystyle{abbrv}
\bibliography{main}
\end{document}
